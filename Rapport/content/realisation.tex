\sectionnn{Introduction réalisation}
Comme ça a été mentionné dans l’état de l’art la visualisation des grands jeux de donnée souffre de cette malédiction de la dimensionalité. Le problème des projections dans ce cas précis n’est pas celui de projeter les données, mais celui de les projeter fidèlement. 
C’est-à-dire, la difficulté pour la méthode à respecter le rapport des distances des points entre le jeu de donné original (à n-dimensions) et la projection finale (qui réduit ce dernier à deux ou 3 dimensions). 
Plus le nombre de dimensions augmente, plus il devient compliqué de respecter les distances après réduction. 
\newline
Le choix d’une méthode de projection par rapport à une autre dépend d'autant plus de la sémantique et de la structure des données. Ce choix dépendra aussi de ce que l’on souhaite observer : voulons-nous mettre en valeur les classes ? Voulons-nous détecter des valeurs aberrantes? 
En fonction de cela, le choix de la méthode et de son paramétrage peut grandement différer.
\smallskip
Une fois le jeu de données projeté, il faut s’assurer de la qualité de cette projection.
Pour cela nous disposons de nombreux critères qui peuvent donner un indice sur le clustering, outlying et la densité de la projection. Certains critères sont même propres à certaines méthodes(comme le projection score pour l'ACP).
Encore une fois, en fonction de ce que nous voulons observer certains critères seront plus utiles que d'autres. De plus, ces critères ne font pas tout : d’une certaine façon, une projection se doit d’être à la fois la plus compréhensible par l’Homme et la plus fidèle possible au jeu de données qu’elle représente.


\medskip
C'est dans ce but, que dans la partie réalisation de cette UE, nous souhaitons mettre en place une évaluation des différentes méthodes de projections citées.
Pour faire cela, nous envisageons de procéder de la façon suivante: \newline
Au vu de nos nombreuses lectures et  des résultats sur notre jeu de données tests (Globe), nous avons déduit que la méthode ClassNeRV est la plus performante et donne les résultats les plus satisfaisants. 
Par conséquent, nous souhaitons effectuer un étalonnage des différentes autres méthodes de projection, en l'utilisant en tant que référentiel. \newline
Notre idée est donc la suivante : lorsqu'un jeu de donnée sera projeté avec une autre méthode de projection, 
le résultat obtenu sera comparé à la projection de ce même jeu de donnée obtenue via ClassNerV.
Puis, en fonction des divergences entre ces deux jeux de données projetés, certains biais seront mesurés et affichés.
De plus dans l'objectif d'être le plus précis possible, nous envisageons également de mesurer les différents critères de la projection(clustering, outlying, forme...) et de croiser les résultats obtenus à ceux de la comparaison des deux projections.
Nous pensons aussi qu'il serait judicieux de mettre en place un système de "scoring", qui permettrait de rendre plus intuitive l'appréhension des résultats relatifs à la qualité de la visualisation. 
\newline 
Le but de cette démarche est de proposer un outil capable de comparer sa méthode de projection à d'autres via un tableau de scoring pour se rendre compte de la pertinence et de la fidélité des diverses méthode de projection.

\medskip


\section{Implémentation}

\subsection{Méthodologie} 
Nous avons donc mis en place un programme avec un système de score qui classe l’efficacité des différentes méthodes de réduction/projection pour un même jeu de données. 
C’est-à-dire que le jeu de donnée va être projeté une fois par chaque algorithme, puis le résultat sera évalué.
Pour mesurer ces résultats nous avons utilisé différents critères issus de la librairie R « clusterCrit » et scikit-learn.
ClusterCrit est une librairie R qui fournit une liste de critéres permettant d'attester de la qualité interne des clusters et une liste de critères qui permet 
de mesurer la similarité entre deux partitions. Ces critères prennent seulement en compte la répartition des points dans les différents cluster et ne permettent pas de 
mesurer la qualité de la distribution.

\subsection{Jeux de données utilisés}
Pour comparer les différentes méthodes nous avons créer plusieurs jeux de donnés qui présentent les caractéristiques suivantes :
\begin{itemize}
    \item deux clusters superposés avec un seul(Figure 5)
    \item trois clusters bien distincts(Figure 6)
    \item trois clusters superposés(Figure 7)
    \item un cluster entouré de quelques outliers(Figure 8)
    \item trois cluster entourés d’outliers(Figure 9)    
\end{itemize}
Pour les créer nous avons utilisé la méthode \textit{makeblobs} de la librairie \textit{sklearn}. Les paramètres fixes sont :
\begin{itemize}
    \item \textit{n samples} qui représente le nombre de points total qui sera également réparti entre les différents clusters. Nous l'avons fixé à vingt-mille.
    \item \textit{n features} qui représente le nombre de variables pour chaque échantillon. Nous l'avons fixé à vingt-et-un.
\end{itemize}
Nous avons fait varier les paramètres suivants pour obtenir la répartition souhaitée : 
\begin{itemize}
    \item \textit{centers} qui détermine le nombre de centres à génerer
    \item \textit{cluster std} qui détermine l'écart type des clusters
    \item \textit{random state} qui détermine un nombre aléatoire pour la crétion du jeu de données.
\end{itemize}


\begin{center}
    \begin{figure}[ht!]
        \centering
        
        \includegraphics[width=6cm, keepaspectratio]{imports/three_distincts.png}
        
        \caption{Trois clusters distincts}
    \end{figure}
\end{center}

\begin{center}
    \begin{figure}[ht!]
        \centering
        
        \includegraphics[width=6cm, keepaspectratio]{imports/three_overlaped.png}
        
        \caption{Trois clusters se chevauchant}
    \end{figure}
\end{center}

\begin{center}
    \begin{figure}[ht!]
        \centering
        
        \includegraphics[width=6cm, keepaspectratio]{imports/three_distincts_plus_outliers.png}
        
        \caption{Trois cluster distincs avec des outliers}
    \end{figure}
\end{center}

\begin{center}
    \begin{figure}[ht!]
        \centering
        
        \includegraphics[width=6cm, keepaspectratio]{imports/one_plus_outliers.png}
        
        \caption{Un cluster avec plusieurs outliers}
    \end{figure}
\end{center}


\begin{center}
    \begin{figure}[ht!]
        \centering
        
        \includegraphics[width=6cm, keepaspectratio]{imports/two_overlaped_one_alone.png}
        
        \caption{Deux clusters se chevauchant et un cluster seul}
    \end{figure}
\end{center}




\subsection{Fonctionnement du programme}
La librairie \textit{rpy2} est utilisée pour pouvoir charger et utiliser des librairies R sous Python. Comme précisé plus haut
nous appliquons cela à la librairie \textit{clusterCrit}. Et appelerons tous les critéres d'intérêts qui sont listé dans la variable globale \textit{crit}.

La fonction \textit{calculate} va utiliser chaque méthode de réduction sur le jeu de donnée, le resultat de cette réduction sera stockée dans
la variable \textit{p2} à laquelle on appliquera la fonction \textit{IntVector} qui permet de convertir un tableau python en vecteur utilisable sous R. 
Comme éléments de comparaison, on applique Kmean sur le jeu de donnée originel. ce qui permet de partitionner le jeu de données en k groupes. 
Dans notre cas k = 3.

Une fois ces deux opérations effectuées, il y a deux variables \textit{rds} (qui représente le résultat de la méthode de réduction), et \textit{rdsOriginal} (qui représente le jeu de
donnée original après kmean.) Pour comparer, ces deux jeux de données nous les passons en paramètre de la fonction \textit{metricsCalcul}.
Celle-ci va itérer à travers tous les critèrs de la liste \textit{crit} et appliquer l'évaluation de chaque critères.

\subsection{Expérimentation}
Les résultats seront stockés sous la forme d'un tableau comprenant les résultats de chaque méthode pour chaque jeux de données. 
Voici un tableau contenant la valeur minimum et Maximum de chaque critères.

\section{Analyse des résultats}

\subsection{L'index de Czekanowski Dice}
On peut voir que pour les trois jeux de données c'est la PCA qui a les meilleurs résultats pour les trois jeux de données.
Parmis les trois jeux de données c'est le premier (deux clusters se chevauchant et un seul) avec lequel on retrouve de meilleur résultat.

\subsection{L'index de Folkes Mallow}
On peut voir que pour les trois jeux de données c'est la PCA qui a les meilleurs résultats pour les trois jeux de données.
Parmis les trois jeux de données c'est le premier (deux clusters se chevauchant et un seul) avec lequel on retrouve de meilleur résultat.

\subsection{L'index de Hubert}

\subsection{L'index de Jaccard}
On peut voir que pour les trois jeux de données c'est la PCA qui a les meilleurs résultats pour les trois jeux de données.
Parmis les trois jeux de données c'est le premier (deux clusters se chevauchant et un seul) avec lequel on retrouve de meilleur résultat.

\subsection{L'index de Kulczynski}
On peut voir que pour les trois jeux de données c'est la PCA qui a les meilleurs résultats pour les trois jeux de données.
Parmis les trois jeux de données c'est le premier (deux clusters se chevauchant et un seul) avec lequel on retrouve de meilleur résultat.
On peut noter que pour le troisième jeu de données (trois clusters se chevauchant) toutes les méthodes de réduction ont la même efficacité : 
elles ont toutes les trois eu un score de 0.18.


\subsection{L'index de McNemar}
On peut voir que pour les trois jeux de données c'est le tSNE qui a les meilleurs résultats pour les trois jeux de données.
Parmis les trois jeux de données c'est le premier (deux clusters se chevauchant et un seul) avec lequel on retrouve de meilleur résultat.
Néanmoins, son score(394.03) est très proche du second jeux de données(393.52).

\subsection{L'index de Jaccard}
On peut voir que pour les trois jeux de données c'est la PCA qui a les meilleurs résultats pour les trois jeux de données.
Parmis les trois jeux de données c'est le premier (deux clusters se chevauchant et un seul) avec lequel on retrouve de meilleur résultat.