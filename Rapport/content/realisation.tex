Comme ça a été mentionné dans l’état de l’art la visualisation des grands jeux de donnée souffre de cette malédiction de la dimensionalité. 
Le problème des projections dans ce cas précis n’est pas celui de projeter les données, mais celui de les projeter fidèlement. 
C’est-à-dire, la difficulté pour la méthode à respecter le rapport des distances des points entre le jeu de donné original (à n-dimensions) et la projection finale (qui réduit ce dernier à deux ou 3 dimensions). 
Plus le nombre de dimensions augmente, plus il devient compliqué de respecter les distances après réduction. // Puis suite du paragraphe.

\medskip

Méthodologie : 
Nous avons donc mis en place un programme avec un système de score qui classe l’efficacité des différentes méthodes de réduction/projection pour un même jeu de donné. 
C’est-à-dire que le jeu de donnée va passer une fois dans chaque algorithme, puis le résultat sera mesuré.
Pour mesurer ces résultats nous avons utilisé différents critères issus de la librairie R « clusterCrit » et scikit-learn.
ClusterCrit est une librairie R qui fournit une liste de critéres permettant d'attester de la qualité interne des clusters et une liste de critères qui permet 
de mesurer la similarité entre deux partitions. Ces critères prennent seulement en compte la répartition des points dans les différents cluster et ne permettent pas de 
mesurer la qualité de la distribution.


